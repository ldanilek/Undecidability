\subsection{$\Delta \vdash H \implies$ machine halts on n}

We know that $\Delta$ is true under our interpretation.  Therefore, $H$ must also be true under this interpretation.
Since $H$ is true if and only if the machine halts on $n$, $\Delta \vdash H \implies$ machine halts on n.

\subsection{machine halts on n $\implies \Delta \vdash H$}

To start out, we introduce a simple convention about negative numbers. Recall that in our definition of the Turing Machine, the tape extended infinitely in both directions, so we had both positive and negative tape locations.

We will write 

\[tQ_ip = \piecewise{ \z^{(t)}Q_i\z^{(p)} & \text{if }p \ge 0 \\ 
 \exists z (\z^{(t)}Q_iz \land z^{(q)} = \z) & \text{if } p < 0}\]

with similar formulas for $xS_jp$ and $y \ne p$.

Equipped with this notation, we can now introduce a new sentence, called a \textit{description of time $s$} that states exactly what the machine looks like at time s. We do this in the obvious way, by specifying

\begin{itemize}
\item the state of the machine
\item the symbols on each square of tape
\item the square the machine is currently scanning
\end{itemize}

In general, such a sentence has the form
\comment{ % using the verbose 0^(p) notation
\begin{eqnarray*}
\z^{(s)}Q_i\z^{(p)} \land \z^{(s)} S_{j_1} \z^{(p_1)} \land \dots\land \z^{(s)} S_j \z^{(p)} \land \dots \land \z^{(s)} S_{j_v}\z^{(p_r)} \land \\
\forall y[(y\ne \z^{(p_1)}\land\dots\land y\ne \z^{(p)} \land \dots \land y\ne \z^{(p_r)} ) \rightarrow \z^{(s)} S_0y]
\end{eqnarray*}}
\begin{eqnarray*}\label{descAtTime}
sQ_ip \land s S_{j_1} p_1 \land \dots\land s S_j p \land \dots \land s S_{j_v}p_r \land \\
\forall y[(y\ne p_1\land\dots\land y\ne p \land \dots \land y\ne p_r ) \rightarrow s S_0y]
\end{eqnarray*}
where $p_1, \dots , p, \dots, p_r$ is an increasing sequence of integers (and $p$ may be $p_1$ or $p_r$)

Note that in the case of the starting conditions of the machine, sentence 10.4 (which we included in $\Delta$) is exactly the description of time 0, under this new type of sentence formulation.




Now, to the proof.
We have machine halts on n.
This implies that at some time s, the machine is in state $q_i$ scanning square number p on which the symbol $s_j$ occurs, but there is no entry for $q_i$, $s_j$ in the machine table.
Consider the description of time s. We would write it as 
\[sQ_ip \land \dots \land sS_jp \land \dots \land \forall y [(\dots)]\]
this is a highly abbreviated form of the sentence, but notice the parts that we have included:
	the conjunction of many terms with the terms
	$sQ_ip \land sS_jp$

Let us suppose -- and this is a big supposition, one that will take a decent amount of proving below -- that $\Delta$ implies this description is true. Then we have 
\[sQ_ip \land \dots \land sS_jp \land \dots \land \forall y [(\dots)] = 1 \tag{$\star$}\]
and in particular
\[sQ_ip \land sS_jp = 1\]

In other words, 
\[\exists t \exists x (tQ_ix \land tS_jx) = 1\]
where $Q_i$ and $S_j$ are not in the machine table.
Notice: this last sentence is exactly one of the or clauses of H. If this last sentence is true, it implies H.

In other words, we now have 
halting description $(\star)\implies H$.\\
All there is now left to do is to show that $\Delta \implies (\star)$ if the machine halts.
We will prove this inductively, showing that at every step of the Turing machine's operation, the description of its state is implied by $\Delta$.


\subsubsection{Induction}
Now, we only need to show that for all non-negative $t$ that if the machine has not halted before time $t$, then $\Delta$ implies some description of time $t$. We will prove this by induction. Specifically, what we will prove is for all times $t$, then if $M$ does not halt before time $t$, then $\Delta \vdash M_t$ where $M_t$ is a description of the machine at time $t$ (equation~\ref{descAtTime}). This induction will show $\Delta$ implies the machine is running exactly when it actually is.

\noindent\textbf{Base case:} $t=0$. $\Delta$ contains, and thus implies, $M_0$ (equation~\ref{descOfZero}), which is a description of time $0$.

\noindent\textbf{Inductive step:} Now, we suppose our induction hypothesis, namely that $\Delta\vdash M_t$ for all $t \leq s$ for some fixed $s$. We additionally assume that the machine has not yet halted at time $s$. Now, we must show that $\Delta\vdash M_{s+1}$ implies some description of time $s+1$.

We already know that $\I$ is a model of $\Delta$, so (\ref{descAtTime}) is true in $\I$. Therefore at time $s$, the machine is in state $q_i$, scanning symbol $S_j$ from square $p$. Since the machine does not halt at $s$, at least one of the following edges must exist:
\begin{enumerate}
\item \igEdge{i}{S_j:S_k}{m}
\item \igEdge{i}{S_j:R}{m}
\item \igEdge{i}{S_j:L}{m}
\end{enumerate}
Here, we will prove only the first case, and leave the latter two as exercises. They are equally wordy and very similar. So, in the first case, one of the sentences in $\Delta$ must be (by construction, equation~\ref{writeTapeEdge}):
\[\forall t\forall x\forall y: \left\{ \left[t Q_i x \land t S_j x\right] \to \underbrace{\left[t' Q_m x \land t' S_k x \land (y \neq x \to ((t S_0 y \to t' S_0 y) \land \cdots \land (t S_r y \to t' S_r y)))\right]}_{\text{this part implies a description of time s+1}} \right\}\]
Together with (\ref{successorOfOneOther}) (each integer is successor of exactly one other), (\ref{additionWorks}) (numbers are distinct, successors behave like addition), and (\ref{descAtTime}) (form of description of time $s$), this implies:
\[\forall y: \left\{ \left[s Q_i p \land s S_j p\right] \to \left[s' Q_m p \land s' S_k p \land (y \neq p \to ((s S_0 y \to s' S_0 y) \land \cdots \land (s S_r y \to s' S_r y)))\right] \right\}\]
\[s' Q_m p \land s' S_k p \land \forall y: \left\{ (y \neq p \to ((s S_0 y \to s' S_0 y) \land \cdots \land (s S_r y \to s' S_r y)))\right\}\]
From this, we know that if the machine has visited positions $p_1,\ldots,p,\ldots,p_v$ then each of those positions has some symbol on it that remains unchanged (except for position $p$) and all outside positions still contain the default symbol $S_0$.
\begin{equation}
\begin{split}
\z^{(s+1)} Q_m \z^{(p)} \land \left( \z^{(s+1)} S_{j_1} \z^{(p_1)} \land \cdots \land \z^{(s+1)} S_k \z^{(p)} \land \cdots \land \z^{(s+1)} S_{j_v} \z^{(p_v)}\right) \\
\land \left(\forall y \left[ (y \neq \z^{(p_1)} \land \cdots \land y \neq \z^{(p_v)}) \to \z^{(s+1)} S_0 y \right]\right)
\end{split}
\end{equation}
In all three cases, $\Delta$ implies a description of time $s+1$ and this completes the proof of the undecidability of first-order logic.