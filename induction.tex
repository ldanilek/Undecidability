Now, we only need to show that for all non-negative $s$ that if the machine has not halted before time $s$, then $\Delta$ implies some description of time $s$. We will prove this by induction.

\noindent\textbf{Base case:} $s=0$. $\Delta$ contains, and thus implies, $(10.4)$, which is a description of time $0$.

\noindent\textbf{Inductive step:} Suppose $\Delta$ implies some description of $\I$ for all times before $s$ and that the machine has not halted before time $s+1$. Now, we must show that $\Delta$ implies some description of time $s+1$.

We already know that $\I$ is a model of $\Delta$, so $(10.8)$ is true in $\I$. Therefore at time $s$, the machine is in state $q_i$, scanning symbol $S_j$ from square $p$. Since the machine does not halt at $s$, at least one of the following edges must exist:
\begin{enumerate}
\item \igEdge{i}{S_j:S_k}{m}
\item \igEdge{i}{S_j:R}{m}
\item \igEdge{i}{S_j:L}{m}
\end{enumerate}

Here, we will show only the first case, and leave the latter two as exercises. They are equally wordy and very similar. So, in the first case, one of the sentences in $\Delta$ must be (by construction):
$$\forall t\forall x\forall y: \left\{ \left[t Q_i x \land t S_j x\right] \to \underbrace{\left[t Q_m x \land t S_k x \land (y \neq x \to ((t S_0 y \to t' S_0 y) \land \cdots \land (t S_r y \to t' S_r y)))\right]}_{\text{this part implies a description of time s+1}} \right\}$$
Together with $(10.5)$ (each integer is successor of exactly one other), $(10.6)$ (numbers are distinct, successors behave like addition), and $(10.8)$ (form of description of time $s$), this implies:
$$\z^{(s+1)} Q_m \z^{(p)} \land \left( \z^{(s+1)} Q_{j_1} \z^{(p_1)} \land \cdots \land \z^{(s+1)} Q_{j_v} \z^{(p_v)}\right) \land \left(\forall y \left[ (y \neq \z^{(p_1)} \land \cdots \land y \neq \z^{(p_v)}) \to \z^{(s+1)} S_0 y \right]\right)$$
In all three cases, $\Delta$ implies a description of time $s+1$ and this completes the proof of the undecidability of first-order logic.